\documentclass[11pt,a4paper]{jsarticle}
\usepackage{amsmath,amssymb}
\usepackage{newtxtext,newtxmath}
\usepackage[dvipdfmx]{graphicx}
\usepackage{listings}
\lstset{%
 language={C++},
 % backgroundcolor={\color[gray]{.95}},%
 tabsize=2, % tab space width
 showstringspaces=false, % don't mark spaces in strings
 basicstyle={\ttfamily},%
 %identifierstyle={\small},%
 commentstyle={\itshape},%
 keywordstyle={\bfseries},%
 %ndkeywordstyle={\small},%
 stringstyle={\ttfamily},
 %frame={tb},
 breaklines=true,
 columns=[l]{fullflexible},%
 % numbers=left,%
 % numberstyle={\small},%
 xrightmargin=0zw,%
 %xleftmargin=3zw,%
 stepnumber=1,
 numbersep=1zw,%
 lineskip=-0.5ex%
}

\title{レポート第2回 \\
  「情報数理科学2 5/7出題課題」}
\author{氏 名: 林橘平\\
        学際科学科 総合情報コース3年 \\
        学生証番号:08-192025 \\
        E-mail: hk1338h0401@gmail.com}
\date{\today}
%
\begin{document}
\maketitle
%
\section{課題1 - Exercise1}
\subsection{プログラムリスト(MyArrayStack.h)}
*配布されたものとの変更点は90 〜 111行目。
\label{sec:prog-list1}
% プログラムは 以下のように \begin{lstlisting} ... \end{lstlisting} を使って表示する.
% また,行番号をつけるときは
% \begin{lstlisting}[numbers=left]
% のようにオプションをつけよ.

\begin{lstlisting}[numbers=left,numberstyle=\ttfamily,xleftmargin=2zw]

/*
 * MyArrayStack.h
 *
 *  Created on: 2011-11-23
 *      Author: morin
 */

#ifndef MYArrayStack_H_
#define MYArrayStack_H_
#include "array.h"
#include "utils.h"

namespace ods {

template<class T>
class DualArrayDeque;

template<class T>
class MyArrayStack {
protected:
	friend class DualArrayDeque<T>;
	array<T> a;
	int n;
	virtual void resize();
public:
	MyArrayStack();
	virtual ~MyArrayStack();
	int size();
	T get(int i);
	T set(int i, T x);
	virtual void add(int i, T x);
	virtual void addAll1(int i, MyArrayStack<T>& x);
	virtual void addAll2(int i, MyArrayStack<T>& x);
	virtual void add(T x) { add(size(), x); }
	virtual T remove(int i);
	virtual void clear();
};

template<class T> inline
int MyArrayStack<T>::size() {
	return n;
}

template<class T> inline
T MyArrayStack<T>::get(int i) {
	return a[i];
}

template<class T> inline
T MyArrayStack<T>::set(int i, T x) {
	T y = a[i];
	a[i] = x;
	return y;
}

template<class T>
void MyArrayStack<T>::clear() {
	n = 0;
	array<T> b(1);
	a.swap(b);
}

template <class T>
MyArrayStack<T>::MyArrayStack() : a(1) {
	n = 0;
}

template<class T>
MyArrayStack<T>::~MyArrayStack() {
}


template<class T>
void MyArrayStack<T>::resize() {
	array<T> b(max(2 * n, 1));
	for (int i = 0; i < n; i++)
		b[i] = a[i];
	a.swap(b);
}

template<class T>
void MyArrayStack<T>::add(int i, T x) {
	if (n + 1 > a.length) resize();
	for (int j = n; j > i; j--)
		a[j] = a[j - 1];
	a[i] = x;
	n++;
}

template<class T>
void MyArrayStack<T>::addAll1(int i, MyArrayStack<T>& x) {
	for (int j = 0; j < x.size(); j++){
		add(i + j, x.get(j));
	}
}

template<class T>
void MyArrayStack<T>::addAll2(int i, MyArrayStack<T>& x) {
	if (a.length < n + x.size()) {
		array<T> b(max(n + x.size(), 1));
		for (int j = 0; j < n; j++) b[j] = a[j];
	a.swap(b);
	}
	for (int j = i; j < n; j++) { 
		a[j + x.size()] = a[j];
	}
	for (int j = i; j < i + x.size(); j++){
		a[j] = x.get(j - i);
	}
	n += x.size() ;
}


template<class T>
T MyArrayStack<T>::remove(int i) {
    T x = a[i];
	for (int j = i; j < n - 1; j++)
		a[j] = a[j + 1];
	n--;
	if (a.length >= 3 * n) resize();
	return x;
}

} /* namespace ods */

#endif /* MyArrayStack_H_ */
\end{lstlisting}
%
\subsection{プログラムリスト(ex2\_1.cpp)}

\label{sec:prog-list2}
\begin{lstlisting}[numbers=left,numberstyle=\ttfamily,xleftmargin=2zw]
#include "MyArrayStack.h"
#include <iostream>

uint64_t cputime(){
  unsigned int ax,dx;
  asm volatile("rdtsc\nmovl %%eax,%0\nmovl %%edx,%1":"=g"(ax),"=g"(dx): :"eax","edx");
  return (uint64_t(dx)<<32)+uint64_t(ax);
}

int main(){
  int n = 20;
  ods::MyArrayStack<int> as, bs, cs;
  for (int i = 0; i < n; i++){
    as.add(i, i); 
    cs.add(i, i);
  }
  for (int i = 0; i < n * 4; i++){
    bs.add(i, 1000 - i); 
  }
  uint64_t cpu1 = cputime();
  cs.addAll1(10,bs);
  uint64_t cpu2 = cputime();
  as.addAll2(10,bs);
  uint64_t cpu3 = cputime();
  std::cout << "addAll1 time " << cpu2 - cpu1 << std::endl;
  std::cout << "addAll2 time " << cpu3 - cpu2 << std::endl;
  for (int i = 0; i < as.size(); i++){
		std::cout << "as[" << i << "]=" <<as.get(i) << std::endl; 		
	}

  return 0;
}

\end{lstlisting}

\subsection{実行結果}
\label{sec:results1}

\begin{quote}           % 実行結果は \begin{quote} で字下げする
\begin{verbatim}
コンパイラはg++ 、自分のMacターミナル上でコマンドを実行した。
(base) MBP:cpp hayashikippei$ g++ -o  ex2_1   ex2_1.cpp
(base) MBP:cpp hayashikippei$ ./ex2_1 > ex2_1.txt
ex2\_1.txtファイルの中身

addAll1 time 46067
addAll2 time 6903
as[0]=0
as[1]=1
as[2]=2
as[3]=3
as[4]=4
as[5]=5
as[6]=6
as[7]=7
as[8]=8
as[9]=9
as[10]=1000
as[11]=999
as[12]=998
as[13]=997
as[14]=996
as[15]=995
as[16]=994
as[17]=993
as[18]=992
as[19]=991
as[20]=990
as[21]=989
as[22]=988
as[23]=987
as[24]=986
as[25]=985
as[26]=984
as[27]=983
as[28]=982
as[29]=981
as[30]=980
as[31]=979
as[32]=978
as[33]=977
as[34]=976
as[35]=975
as[36]=974
as[37]=973
as[38]=972
as[39]=971
as[40]=970
as[41]=969
as[42]=968
as[43]=967
as[44]=966
as[45]=965
as[46]=964
as[47]=963
as[48]=962
as[49]=961
as[50]=960
as[51]=959
as[52]=958
as[53]=957
as[54]=956
as[55]=955
as[56]=954
as[57]=953
as[58]=952
as[59]=951
as[60]=950
as[61]=949
as[62]=948
as[63]=947
as[64]=946
as[65]=945
as[66]=944
as[67]=943
as[68]=942
as[69]=941
as[70]=940
as[71]=939
as[72]=938
as[73]=937
as[74]=936
as[75]=935
as[76]=934
as[77]=933
as[78]=932
as[79]=931
as[80]=930
as[81]=929
as[82]=928
as[83]=927
as[84]=926
as[85]=925
as[86]=924
as[87]=923
as[88]=922
as[89]=921
as[90]=10
as[91]=11
as[92]=12
as[93]=13
as[94]=14
as[95]=15
as[96]=16
as[97]=17
as[98]=18
as[99]=19

\end{verbatim}
\end{quote}
%
\subsection{考察}
\begin{verbatim}
配布された、MyArrayStack.h,とex2_1.cppに手を加えた。効率が悪い実装として公開されていたaddAll関数をaddAll1、自分で実装した方をaddAll2とした。x.size() = mとする。1回のa[x] = a[y]と配列の要素に数値を代入する作業をコピーと呼ぶ。
addAll1では、add(i, x)をm回繰り返すことでaddAllを実装している。ここでは、a[i] ~ a[n-1]の要素をadd(i,x )を実行する度に1ずつ移動させている。その際m (n - i) 回コピーをしているが、ここで無駄が生じている。これらの要素は最終的にa[i + m] ~ a[n+m-1]に移動するので、ここに直接コピーしてやれば良い。この場合だとコピーは (n - i)回で済む。配列xの要素を全てコピーするのでそれを合わせて考え、resizeにかかる時間を無視すると、実行時間はO(m(n + 1 - i))である。
addAll2では上記の無駄を改善したプログラムである。まず配列aのサイズがn + mより小さかったらresizeと同じようにしてサイズをn + mにする。そして、元の配列aのa[i] ~ a[n-1]の要素を添字をmだけ後ろに移動させる。それからxの要素x[0] ~ x[m-1]をa[i] ~ a[i + m -1]に順番にコピーして、addAllの作業が完成する。出力されたex2_1.txtのasの要素を見ても、正しく実行されていることがわかる。resizeの実行時間を無視するとaddAll2の実行時間はO( m + n - i )である。
O( m + n - i ) < O(m(n + 1 - i) = O( m + m(n  - i)) であるので、理論上addAll2で効率化に成功している。
ex2_1.cppでは全く同じ配列as, csを用意して、cs.addAll1(10,bs)にかかった時間をcount1, as.addAll2(10,bs)にかかった時間をcount2としてex2_1.txtに出力した。
count2はcount1の1/10程度になっており、実際に効率化できていることも確認できた。
\end{verbatim}
%
\section{課題2 -Exercise2}
\subsection{プログラムリスト(ArrayStack.h)}
\label{sec:prog-list2}

\begin{lstlisting}[numbers=left,numberstyle=\ttfamily,xleftmargin=2zw]
配布されたものとの変更点は98 〜106行目。
/*
 * ArrayStack.h
 *
 *  Created on: 2011-11-23
 *      Author: morin
 */

#ifndef ARRAYSTACK_H_
#define ARRAYSTACK_H_
#include "array.h"
#include "utils.h"

namespace ods {

template<class T>
class DualArrayDeque;

template<class T>
class ArrayStack {
protected:
	friend class DualArrayDeque<T>;
	array<T> a;
	int n;
	virtual void resize();
public:
	ArrayStack();
	virtual ~ArrayStack();
	int size();
	T get(int i);
	T set(int i, T x);
	virtual void add(int i, T x);
	virtual void add(T x) { add(size(), x); }
	virtual T remove(int i);
	virtual T remove_random();
	virtual void clear();
};

template<class T> inline
int ArrayStack<T>::size() {
	return n;
}

template<class T> inline
T ArrayStack<T>::get(int i) {
	return a[i];
}

template<class T> inline
T ArrayStack<T>::set(int i, T x) {
	T y = a[i];
	a[i] = x;
	return y;
}

template<class T>
void ArrayStack<T>::clear() {
	n = 0;
	array<T> b(1);
	a.swap(b);
}

template <class T>
ArrayStack<T>::ArrayStack() : a(1) {
	n = 0;
}

template<class T>
ArrayStack<T>::~ArrayStack() {
}

template<class T>
void ArrayStack<T>::resize() {
	array<T> b(max(2 * n, 1));
	for (int i = 0; i < n; i++)
		b[i] = a[i];
	a.swap(b);
}

template<class T>
void ArrayStack<T>::add(int i, T x) {
	if (n + 1 > a.length) resize();
	for (int j = n; j > i; j--)
		a[j] = a[j - 1];
	a[i] = x;
	n++;
}

template<class T>
T ArrayStack<T>::remove(int i) {
	T x = a[i];
	for (int j = i; j < n - 1; j++)
		a[j] = a[j + 1];
	n--;
	if (a.length >= 3 * n) resize();
	return x;
}

template<class T> 
T ArrayStack<T>::remove_random() {
// replace this function
  int i = rand() % size();
  T x = a[i];  
	a[i] = a[n-1];
	n--;
	if (a.length >= 3 * n) resize();
	return x;
}

} /* namespace ods */

#endif /* ARRAYSTACK_H_ */
\end{lstlisting}

\subsection{プログラムリスト(MyRandomQueue.h)}
配布されたものとの変更点は35 〜 39行目。
\label{sec:prog-list2}
\begin{lstlisting}[numbers=left,numberstyle=\ttfamily,xleftmargin=2zw]
#include "ArrayStack.h"

template<class T>
class MyRandomQueue {
protected:
  ods::ArrayStack<T> as;
public:
  MyRandomQueue();
  int size();
  void add(T x);
  T remove();
  T remove_random();
};

template <class T>
MyRandomQueue<T>::MyRandomQueue() : as() {
}

template<class T> inline
int MyRandomQueue<T>::size() {
  return as.size();
}

template<class T> inline
void MyRandomQueue<T>::add(T x) {
  as.add(x);
}

template<class T> inline
T MyRandomQueue<T>::remove() {
// replace this function
  return as.remove(as.size() - 1);
}

template<class T> inline
T MyRandomQueue<T>::remove_random() {
// replace this function
  return as.remove_random();
}


\end{lstlisting}

\subsection{プログラムリスト(ex2\_2.cpp)}
\label{sec:prog-list2}
\begin{lstlisting}[numbers=left,numberstyle=\ttfamily,xleftmargin=2zw]
#include "SlowRandomQueue.h"
#include "MyRandomQueue.h"
#include <iostream>

uint64_t cputime(){
  unsigned int ax,dx;
  asm volatile("rdtsc\nmovl %%eax,%0\nmovl %%edx,%1":"=g"(ax),"=g"(dx): :"eax","edx");
  return (uint64_t(dx)<<32)+uint64_t(ax);
}

int main(){
  int n = 100;
  {
    SlowRandomQueue<int> as;
    uint64_t cpu1 = cputime();
    for (int i = 0; i < n * 100; i++){
      as.add(i);
    }
    int r = 0;
    for (int i = 0; i < n * 100; i++){
      r += as.remove();
    }
    uint64_t cpu2 = cputime();
    std::cout << "SlowRandomQueue time=" << (cpu2 - cpu1) << std::endl;
    for (int i = 0; i < n; i++){
      as.add(i);
    }
    for (int i = 0; i < n; i++){
      std::cout << as.remove() << ",";
    }
    std::cout << std::endl;
  }
  {
    MyRandomQueue<int> as;
    uint64_t cpu1 = cputime();
    for (int i = 0; i < n * 100; i++){
      as.add(i);
    }
    int r = 0;
    for (int i = 0; i < n * 100; i++){
      r += as.remove_random();
    }
    uint64_t cpu2 = cputime();
    std::cout << "MyRandomQueue time=" << (cpu2 - cpu1) << std::endl;
    for (int i = 0; i < n; i++){
      as.add(i);
    }
    for (int i = 0; i < n; i++){
      std::cout << as.remove_random() << ",";
    }
    std::cout << std::endl;
  }
}

\end{lstlisting}

\subsection{実行結果}
\begin{quote}           % 実行結果は \begin{quote} で字下げする
\begin{verbatim}
コンパイラはg++ 、自分のMacターミナル上でコマンドを実行した。
(base) MBP:cpp hayashikippei$ g++ -o  ex2_2   ex2_2.cpp
(base) MBP:cpp hayashikippei$ ./ex2_2 > ex2_2.txt

ex2\_1.txtファイルの中身

SlowRandomQueue time=491136764
6,66,99,81,44,27,46,83,3,84,97,70,2,62,8,40,16,89,18,43,58,67,41,68,7,91,25,13,59,56,9,4,29,60,77,
54,
78,49,98,95,53,92,17,52,71,88,82,42,24,22,10,79,35,80,90,33,76,15,37,5,21,26,55,47,65,45,14,73,
74,87,28,48,96,64,57,72,20,61,12,85,50,23,39,1,75,94,63,36,0,93,19,32,69,11,38,86,30,51,31,34,
MyRandomQueue time=1508515
69,27,78,80,26,44,30,79,39,38,52,33,76,64,24,77,63,56,25,47,74,20,62,60,34,17,9,91,43,16,0,41,90,
11,84,35,32,86,3,45,61,36,95,59,37,28,85,55,83,19,99,98,75,72,94,2,93,66,73,21,14,96,54,13,15,68,
71,29,22,40,92,18,89,70,4,31,10,5,23,53,6,46,67,82,12,65,42,57,48,97,58,1,51,88,81,87,50,8,49,7,

\end{verbatim}
\end{quote}
%
\subsection{考察}
\begin{verbatim}
配布されたArrayStack.h、MyRandomQueue.h、ex2_2.cppに手を加えた。自分で実装したRandomQueueにおけるremove関数をremove_randomとしている。remove_randomはArrayStack.hで定義している。まず、削除する要素のインデックスをrand関数を使って決め、xにその値を記録する。そしてa[i] = a[n-1]; n--; とすることでa[i]を消去している。RandomQueueにおいて、配列のインデックスを指定して要素にアクセスすることはないため、配列の要素はaddした順番に並んでいる必要はない。そのため、n番目の要素a[n-1]をi番目に持ってきてもよく、1回のコピーでremoveを実現できる。そのため理論上これが最も効率が良い。MyRandomQueue.hではremove_randomをArrayStack.hから呼び出して定義している。
ex2_2.txtの出力を見ると、確かにRandomQueueを実装できており、実行時間はSlowRandomQueue.hの1/100以下であり、確かに効率化できていることが確認できる。
\end{verbatim}
%
\end{document}
